\pdfminorversion=4 % Required for compatibility
\documentclass[9pt]{beamer}
% Option of beamer class:
% handout: for printing
% aspectratio=169: for switching from 4:3 to 16:9 ratio
\usepackage{beamerthemefemto}


%% TITLE PAGE INFORMATION
\title{Relatively short title}
\subtitle{Subtitle giving more details}
\author{Dr. Robin Roche \texorpdfstring{-- \url{robin.roche@femto-st.fr}}{}}
\institute{FEMTO-ST, FCLAB, CNRS, Univ. Bourgogne Franche-Comte, UTBM}
\date{Event name -- Location, date}


% ------------------------------------------------------------

\begin{document}

% ------------------------------------------------------------

% Set title page background
\setbeamertemplate{background}{\titrefemto}

% Display title page with contents
\begin{frame}[plain]
\titlepage
\end{frame}
 
% Set normal page background
\setbeamertemplate{background}{\pagefemto}

% ------------------------------------------------------------

% Display table of contents
\begin{frame}{Outline}
	\tableofcontents[pausesections,subsectionstyle=hide]
\end{frame}

% ------------------------------------------------------------

\begin{frame}{Introduction}

\lipsum[1-2]

\end{frame}

% -----------

\section{First section}

% -----------

\subsection{A subsection title}

% -----------

\begin{frame}{A frame title}

A list of something with sequential animations using overlays:
\begin{itemize}
\item Option A with a cost of \EUR{10} and a voltage of 20.51\phase{10\degree} kV
\item \visible<2->{Option B\footnote<2->{Here is an example footnote} with a footnote and a figure}
\item \visible<3->{Option C with a chemical formula: \ch{2 H2 +  O2 -> 2 H2O}}
\item \visible<4->{Option D with an equation:
\begin{equation}
X(s)=\int _{0}^{\infty }e^{-st}\,dt = \left[ \dfrac{e^{-st}}{-s} \right]_0^\infty = \dfrac{1}{s}
\label{eq:example}
\end{equation}
}
\end{itemize}

\visible<2->{
\vspace*{1em}
\begin{figure}[hbtp]
\centering
\includegraphics[height=7em]{figures/logo_FEMTO-ST.pdf}
\caption{Example of figure caption}
\label{fig:example}
\end{figure}
}

\end{frame}

% -----------

\begin{frame}{Another frame title}

An example of tikz figure:
\vspace*{1em}
\begin{figure}
\scalebox{0.9}{
\begin{tikzpicture}[auto, node distance=1.5cm]
	\node (input) {$V_{\mathrm{ref}}(s)$};
    \node [sum, right of=input, node distance=1.5cm] (sum) {};
    \node [block, right of=sum, node distance=2.5cm] (amplifier) {$\dfrac{K_A}{1+\tau_As}$};
  	\node [below of=amplifier, node distance=0.85cm] (amplifier_txt) {Amplifier};
    \node [block, right of=amplifier, node distance=2.5cm] (exciter) {$\dfrac{K_E}{1+\tau_Es}$};
  	\node [below of=exciter, node distance=0.85cm] (exciter_txt) {Exciter};
    \node [block, right of=exciter, node distance=2.5cm] (generator) {$\dfrac{K_G}{1+\tau_Gs}$};
  	\node [below of=generator, node distance=0.85cm] (generator_txt) {Generator};
    \node [right of=generator, node distance=2.5cm] (output) {$V_t(s)$};
    
    \draw [->] (input) -- (sum);
    \draw [->] (sum) -- node [name=ve_txt] {\scriptsize $V_e$} (amplifier);
    \draw [->] (amplifier) -- node [name=vr_txt] {\scriptsize $V_R$} (exciter);

    \node [below of=ve_txt, node distance=1cm] (dummy) {}; 
    \node [block, below of=vr_txt, node distance=2.1cm, draw=red] (stabilizer) {\textcolor{red}{$\dfrac{K_Fs}{1+\tau_Fs}$}};
  	\node [below of=stabilizer, node distance=0.85cm] (stabilizer_txt) {\textcolor{red}{Stabilizer}};  
    \node [block, below of=exciter, node distance=3.6cm] (sensor) {$\dfrac{K_R}{1+\tau_Rs}$};
  	\node [below of=sensor, node distance=0.85cm] (sensor_txt) {Sensor};   

    \draw [->] (exciter) -- node {\scriptsize $V_F$}  node [pos=0.5,name=ybis] {} (generator);
    \draw [red,->] (ybis) |- (stabilizer);    
    \draw [red,-] (dummy) |- node [pos=0.062,name=yter] {} (stabilizer); 
    \draw [red,->] (yter) -- node[pos=0.65,below=0.085cm] {$-$} (sum);     
    \draw [->] (generator) -- node [pos=0.5,name=y] {} (output);
    \draw [->] (y) |- (sensor);
    \draw [->] (sensor) -| node[pos=0.92] {$-$} (sum);    
\end{tikzpicture}
}
\caption{A tikz block diagram example}
\end{figure}

\end{frame}

% -----------

\subsection{Another subsection title}

% -----------

\begin{frame}{Another frame title}

Another list of something with a pause:
\begin{enumerate}
\item Option D citing references \cite{celik-rser,iet-book}
\pause
\item Option E with a reference to Fig.~\ref{fig:example} and Table~\ref{tab:example}
\item Option F with a \hyperlink{appendixA}{link to another slide} or an \href{https://www.robinroche.com/}{external website}
\item Option G with a ~\hyperlink{appendixA}{\beamergotobutton{button link}}
\end{enumerate}

\vspace*{1em}
\begin{table}
\centering
\caption{Table of greek letters}
\begin{tabular}{|l|c|C{3cm}|}
\hline 
\multicolumn{2}{|c|}{\textbf{Letter and symbol}} & \textbf{Comment}\\
\hline 
Alpha & $\alpha$ & \multirow{2}{*}{Nothing}\\ 
\cline{1-2}
Beta & $\beta$ & \\ 
\hline
Gamma & $\gamma$ & Not much\\ 
\hline 
\end{tabular} 
\label{tab:example}
\end{table}

\end{frame}

% -----------

\begin{frame}{One more frame title}

\begin{block}{Block title}
Lots of things you can say
\end{block}

\begin{redblock}{Block title}
Important things you can say
\end{redblock}

\begin{greenblock}{Block title}
Put an example here
\end{greenblock} 

\vspace*{1.5em}
Another example with two columns:
\vspace*{1em}
\begin{columns}[c]
\column{.48\textwidth}
Contents of the first column with some text and stuff 
\column{.48\textwidth}
Contents of the second column with some text and stuff 
\end{columns}

\end{frame}

% -----------

\begin{frame}[fragile]{Another one again}

Here is an example of Matlab code\footnote{Do not forget the [fragile] option in the frame definition}:
\begin{algorithm}[H]
\begin{lstlisting}[language=Matlab]
% Variables
ka = 10; ke = 1; kg = 1; kf = 2; kr = 1;
tau_a = 0.1; tau_e = 0.4; tau_g = 1.0; tau_f = 0.04; tau_r = 0.05;

% Transfer functions
GA = tf([ka],[tau_a 1]);
GE = tf([ke],[tau_e 1]);
GG = tf([kg],[tau_g 1]);
HF = tf([kf 0],[tau_f 1]);
HR = tf([kr],[tau_r 1]);

% Combine transfer functions
T1 = series(GA,GE);
T2 = feedback(T1,HF);
T3 = series(T2,GG);
T4 = feedback(T3,HR)
\end{lstlisting}
\caption{Example of code}
\label{alg:code}
\end{algorithm}

\end{frame}

% -----------

\begin{frame}[fragile]{And a final one}

Here is another algorithm with pseudocode:
\begin{algorithm}[H]
\begin{algorithmic}[1] % [1]: adds line number each line
\vspace*{0.2em}
\If {$i\geq maxval$}
    \State $i\gets 0$
\Else
    \If {$i+k\leq maxval$}
        \State $i\gets i+k$ \Comment{This is a comment}
    \EndIf
\EndIf
\end{algorithmic}
\caption{Example of algorithm pseudocode}
\label{alg:algo}
\end{algorithm}


\end{frame}

% -----------

\section{Conclusion}

% -----------

\begin{frame}{Conclusion}

\lipsum[1-1]

\end{frame}

% -----------

\begin{frame}[allowframebreaks]
\frametitle{For further reading}

\footnotesize

\begin{thebibliography}{bib}

\bibitem[1]{celik-rser}
Berk Celik, Robin Roche, Siddharth Suryanarayanan, David Bouquain and Abdellatif Miraoui
\newblock {\em Electric energy management in residential areas through coordination of multiple smart homes}.
\newblock Renewable and Sustainable Energy Reviews, vol. 80, pp. 260--275, 2017.

\bibitem[2]{iet-book}
Siddharth Suryanarayanan, Robin Roche and Timothy M. Hansen
\newblock {\em Cyber-Physical-Social Systems and Constructs in Electric Power Engineering}.
\newblock The IET, ISBN: 978-1-84919-936-0, 2016.

\end{thebibliography}
\end{frame}

% -----------

% Switch background
\setbeamertemplate{background}{\videfemto}

% Thank you page
\begin{frame}[plain]

\begin{center}
\Huge Thank you for your attention

\vspace*{1em}
\normalsize \url{robin.roche@femto-st.fr}

\end{center}

\end{frame}

% -----------

\appendix\addtocounter{part}{-1}

% -----------

% Switch background
\setbeamertemplate{background}{\appendixfemto}

% Thank you page
\begin{frame}[plain]

\vspace*{1.5em}
\begin{center}
\Huge Appendices
\end{center}

\end{frame}

% Set normal page background
\setbeamertemplate{background}{\pagefemto}

% -----------

\section{Appendix A}

% -----------

\begin{frame}{Appendix A contents}
\label{appendixA}

\lipsum[1-1]

\end{frame}

% -----------

\section{Appendix B}

% -----------

\begin{frame}{Appendix B contents}

\lipsum[1-1]

\end{frame}

% -----------

\end{document}

